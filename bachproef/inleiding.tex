%%=============================================================================
%% Inleiding
%%=============================================================================

\chapter{\IfLanguageName{dutch}{Inleiding}{Introduction}}
\label{ch:inleiding}

In dit onderdeel zal er al een beter beeld worden geschept over het onderwerp zelf en of dit de moeite waard is om te onderzoeken.

\section{\IfLanguageName{dutch}{Probleemstelling}{Problem Statement}}
\label{sec:probleemstelling}

Wanneer een bedrijf zoals Endare, de opdrachtgever van deze onderzoeksvraag een chatbot ontwikkelt voor een klant, zal er veel werk nodig zijn in het op punt zetten van deze chatbot. Een deel van dit werk zal pas kunnen worden gedaan wanneer de chatbot effectief in gebruik is genomen. Al de input dat gebruikers ingeven waarvan LUIS zelf maar een beetje aan twijfelt zal moeten manueel gelinkt worden aan de juiste intent. Door een antwoord te zoeken op de onderzoeksvraag zal dit werk worden verminderd door een gedeelte van deze inputs automatisch te koppelen aan de juiste intent.

\section{\IfLanguageName{dutch}{Onderzoeksvraag}{Research question}}
\label{sec:onderzoeksvraag}

De onderzoeksvraag 'Kunnen we een chatbot automatisch laten bijleren op basis van het gedrag van gebruikers?', zal in deze bachelorproef onderzocht worden. Kan hier door middel van de input die gebruikers ingeven via de chatbot, deze input automatisch worden gelinkt aan de correcte intent, dus de bot laten bijleren. Hierbij zal ook onderzocht worden in welke mate dit automatisch zal gebeuren en het werk voor het ontwikkelteam verminderen.

\section{\IfLanguageName{dutch}{Onderzoeksdoelstelling}{Research objective}}
\label{sec:onderzoeksdoelstelling}

Dit werk zal een succes zijn als de API van minstens één van de onderzochte frameworks toelaat dat een input van een gebruiker kan opgehaald en gecontroleerd worden om dit dan te linken aan de correcte intent. Er zal een proof of concept worden opgesteld om dit te realiseren.

\section{\IfLanguageName{dutch}{Opzet van deze bachelorproef}{Structure of this bachelor thesis}}
\label{sec:opzet-bachelorproef}

% Het is gebruikelijk aan het einde van de inleiding een overzicht te
% geven van de opbouw van de rest van de tekst. Deze sectie bevat al een aanzet
% die je kan aanvullen/aanpassen in functie van je eigen tekst.

De rest van deze bachelorproef is als volgt opgebouwd:

In Hoofdstuk~\ref{ch:stand-van-zaken} wordt een overzicht gegeven van de stand van zaken binnen het onderzoeksdomein, op basis van een literatuurstudie.

In Hoofdstuk~\ref{ch:methodologie} wordt de methodologie toegelicht en worden de gebruikte onderzoekstechnieken besproken om een antwoord te kunnen formuleren op de onderzoeksvragen.

% TODO: Vul hier aan voor je eigen hoofstukken, één of twee zinnen per hoofdstuk

In Hoofdstuk~\ref{ch:poc} wordt de gemaakte proof of concept besproken. Hierin wordt ook gezien hoe het gekozen framework zal worden opgezet.

In Hoofdstuk~\ref{ch:res} zal de accuraatheid en de efficiëntie van de proof of concept worden getest.

In Hoofdstuk~\ref{ch:conclusie} tenslotte, wordt de conclusie gegeven en een antwoord geformuleerd op de onderzoeksvragen. Daarbij wordt ook een aanzet gegeven voor toekomstig onderzoek binnen dit domein.