%%=============================================================================
%% Voorwoord
%%=============================================================================

\chapter*{\IfLanguageName{dutch}{Woord vooraf}{Preface}}
\label{ch:voorwoord}

%% TODO:
%% Het voorwoord is het enige deel van de bachelorproef waar je vanuit je
%% eigen standpunt (``ik-vorm'') mag schrijven. Je kan hier bv. motiveren
%% waarom jij het onderwerp wil bespreken.
%% Vergeet ook niet te bedanken wie je geholpen/gesteund/... heeft

Het onderwerp van deze bachelorproef kwam ter sprake met nog een aantal andere onderwerpen die Endare, mijn stagebedrijf wou onderzoeken. Dit onderwerp gaande over chatbots sprak mij onmiddellijk aan door de enorme groei van AI en Machine learning. Ook chatbots is een onderdeel hiervan dus leek het mij interessant om hier dieper op in te gaan en dit te onderzoeken. Chatbots zullen meer en meer worden gebruikt in de toekomst. De ontwikkeling van chatbots met AI staan nog niet zo super ver en daardoor wou ik mij daar ook wat meer in verdiepen en een meerwaarde bieden in dit domein.

Ik zou graag de personen willen bedankten die mij hebben geholpen bij het realiseren van deze bachelorproef. Mijn promotor Leen Vuyge voor mij zo goed te gegeleiden tijdens dit onderzoek en altijd tijd vrij te maken om samen te zitten en mij zo goed mogelijk al mijn vragen uit te leggen.

Ook zou ik mijn co-promoter Sander Goossens willen bedanken voor de hulp en de verduidelijking van dit onderzoek. Hij heeft mij geholpen met het zoeken van een onderwerp, het verduidelijken van vragen en problemen die ik had en de tijd voor samen te denken voor een oplossing op de onderzoeksvraag.

Als laatste zou ik mijn ouders en grootvader willen bedanken voor het nalezen en controleren van deze bachelorproef. Ook voor de grote steun tijdens het schrijven van deze.
